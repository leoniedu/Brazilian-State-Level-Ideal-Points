The Statistical Analysis of Electoral Cartels

In this paper I derive a statistical model from a simple theory of electoral coalition (cartel) formation in proportional representation systems. The theory predicts that parties are more likely to form coalitions if they are ideologically close, if the number of seats in dispute is small, or if the parties themselves are small. The statistical model produces estimates of the structural parameters, as well as the ideological location of parties. I apply the model to the Brazilian local elections of 2000, 2004 and 2008.  The party positions estimated from this model are shown to be highly correlated to those measured using the PLIO surveys. 